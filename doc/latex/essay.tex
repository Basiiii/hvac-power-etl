%----------------------------------------------------------------------------------------
%	PACKAGES AND OTHER DOCUMENT CONFIGURATIONS
%----------------------------------------------------------------------------------------

\documentclass[a4paper, 12pt]{article} % Font size (can be 10pt, 11pt or 12pt) and paper size (remove a4paper for US letter paper)
\usepackage[portuguese]{babel}
\usepackage{amsmath}
\usepackage[protrusion=true,expansion=true]{microtype} % Better typography
\usepackage{graphicx} % Required for including pictures
\usepackage{wrapfig} % Allows in-line images
\usepackage{pdflscape} %Allows landscape oriented pages
\usepackage{ulem} % para sublinhar
\usepackage[colorlinks=true, urlcolor=blue]{hyperref}%Required for hyperlink references
\hypersetup{
	colorlinks=true, % Ativa links coloridos em vez de caixas ao redor
	linkcolor=black, % Cor dos links internos
	citecolor=black, % Cor das citações
	urlcolor=black   % Cor dos links externos (Jira, por exemplo)
}

\usepackage{mathpazo} % Use the Palatino font
\usepackage[T1]{fontenc} % Required for accented characters
\linespread{1.05} % Change line spacing here, Palatino benefits from a slight increase by default
\usepackage{float}
\usepackage[backend=bibtex,style=numeric]{biblatex}
\addbibresource{references.bib} % Nome do arquivo de referências
\makeatletter
\renewcommand\@biblabel[1]{\textbf{#1.}} % Change the square brackets for each bibliography item from '[1]' to '1.'
\renewcommand{\@listI}{\itemsep=0pt} % Reduce the space between items in the itemize and enumerate environments and the bibliography

\usepackage{amsmath, amssymb}
\usepackage{graphicx}
\usepackage{hyperref}
\usepackage{geometry}
\usepackage{enumitem}
\usepackage{qrcode}

\renewcommand{\maketitle}{
\begin{titlepage}
\begin{center}
\vspace*{1cm}
%\includegraphics[width=0.35\textwidth]{../images/logo-no-bg.png}\\[1cm] % Logo
{\Huge\textbf{Clima e Consumo HVAC}}\\[0.5cm] % Main Title
{\Large Integração de Sistemas de Informação}\\[2cm] % Subtitle
{\large \textsc{
	Enrique Rodrigues Nº28602}}\\[0.5cm] % Author
{\textit{Instituto Politécnico do Cávado e do Ave}}\\[1.5cm] % Institution
{\large \today} % Date
\vfill
\textbf{Palavras-chave:} Node-RED, Pentaho Kettle, Integração de Dados, IoT, Simulação, HVAC, Modelação Térmica, ETL, Python, pandas, matplotlib
\end{center}
\end{titlepage}
}
\makeatother

%------------------------------------------------------------------------------------
\begin{document}
\maketitle % Print the title section

%----------------------------------------------------------------------------------------
%	ABSTRACT AND KEYWORDS
%----------------------------------------------------------------------------------------

\renewcommand{\abstractname}{Summary} % Uncomment to change the name of the abstract to something else

% \begin{abstract}
% Morbi tempor congue porta. Proin semper, leo vitae faucibus dictum, metus mauris lacinia lorem, ac congue leo felis eu turpis. Sed nec nunc pellentesque, gravida eros at, porttitor ipsum. Praesent consequat urna a lacus lobortis ultrices eget ac metus. In tempus hendrerit rhoncus. Mauris dignissim turpis id sollicitudin lacinia. Praesent libero tellus, fringilla nec ullamcorper at, ultrices id nulla. Phasellus placerat a tellus a malesuada.
% \end{abstract}

% \hspace*{3,6mm}\textit{Keywords:} lorem , ipsum , dolor , sit amet , lectus % Keywords

% \vspace{30pt} % Some vertical space between the abstract and first section

%----------------------------------------------------------------------------------------
%	DOCUMENT BODY
%----------------------------------------------------------------------------------------

\newpage
\renewcommand{\contentsname}{Índice}
\tableofcontents

%------------------------------------------------------------------------------------

%\newpage
%\renewcommand{\listfigurename}{Lista de Figuras}
%\listoffigures

%------------------------------------------------------------------------------------

\newpage
\section{Enquadramento}

O presente trabalho descreve o desenvolvimento de um pipeline de dados completo para recolha, transformação, armazenamento e análise de variáveis ambientais, nomeadamente:
temperatura exterior, temperatura interior simulada e consumo energético de um sistema HVAC.

O sistema é composto por quatro módulos principais:
\begin{itemize}
	\item \textbf{Node-RED:} aquisição de dados e simulação das variáveis ambientais;
	\item \textbf{Pentaho Data Integration (Kettle):} transformação, limpeza, enriquecimento e exportação dos dados;
	\item \textbf{Base de Dados SQLite:} armazenamento estruturado e persistente dos resultados processados;
	\item \textbf{Python (pandas, matplotlib, seaborn):} análise estatística e visualização gráfica dos resultados.
\end{itemize}

%------------------------------------------------------------------------------------

\newpage
\section{Problema}
A gestão eficiente da climatização em edifícios residenciais e comerciais é um desafio crescente, dada a variabilidade das condições exteriores e o impacto direto no consumo energético.
O controlo manual do HVAC ou a falta de monitorização contínua leva a desperdício de energia e a condições de conforto inconsistentes.\

O objetivo deste trabalho é demonstrar um pipeline de dados capaz de:
\begin{itemize}
	\item Recolher temperaturas exteriores via API pública, permitindo monitorização contínua;
	\item Simular a evolução da temperatura interior e o consumo do HVAC, com base em modelos térmicos iterativos;
	\item Processar, validar e enriquecer os dados no Pentaho, incluindo cálculos de eficiência e diferença térmica;
	\item Exportar os resultados para vários formatos — \textbf{CSV, XML e Base de Dados SQLite} — de modo a permitir integração e análise em diferentes plataformas;
	\item Gerar gráficos e análises estatísticas em Python que permitam avaliar padrões de consumo e identificar oportunidades de otimização energética.
\end{itemize}

Desta forma, o pipeline não só replica o comportamento térmico de um edifício como também fornece dados estruturados, persistentes e prontos para análise, permitindo uma avaliação clara da eficiência do sistema HVAC e do impacto das condições ambientais.

%------------------------------------------------------------------------------------

\newpage
\section{Estratégia Utilizada}

\subsection{Aquisição e Simulação (Node-RED)}

Os dados de temperatura exterior são obtidos via API REST.  
A simulação da temperatura interior e do consumo do HVAC é realizada em JavaScript.
Cada sala tem parâmetros térmicos próprios (\(\alpha, \beta\)).
\subsection{Modelação Matemática das Variáveis Simuladas}

Esta secção descreve os modelos utilizados para simular a evolução térmica das divisões e o consumo energético do sistema HVAC.  
Os modelos foram implementados em JavaScript, correndo em ciclo a partir da temperatura anterior e das condições exteriores obtidas por API.  
Os parâmetros $\alpha$ e $\beta$ variam por divisão e representam, respetivamente, o grau de isolamento e a capacidade de resposta do sistema de climatização.

\subsubsection*{Evolução da Temperatura Interior}

A temperatura interior de cada divisão evolui de forma gradual, aproximando-se da temperatura exterior devido às trocas térmicas do edifício e da temperatura de conforto devido à atuação do HVAC.
A equação seguinte modela esta dinâmica:

\[
T_{\text{interior}}(t) = T_{\text{anterior}} + \alpha \cdot (T_{\text{exterior}} - T_{\text{anterior}}) + \beta \cdot (T_{\text{conforto}} - T_{\text{anterior}}) + \epsilon
\]

\textbf{Onde:}
\begin{itemize}[nosep]
	\item $T_{\text{interior}}(t)$: temperatura interior no instante atual;
	\item $T_{\text{anterior}}$: temperatura interior no instante anterior;
	\item $T_{\text{exterior}}$: temperatura exterior (obtida por API);
	\item $T_{\text{conforto}}$: temperatura de conforto (21°C);
	\item $\alpha$: coeficiente de isolamento térmico do edifício;
	\item $\beta$: coeficiente de correção aplicado pelo HVAC;
	\item $\epsilon$: ruído aleatório que introduz variação realista.\\
\end{itemize}

Este modelo é iterativo - a cada ciclo de simulação, a temperatura interior é atualizada em função dos valores anteriores.  
A componente $\alpha$ controla a velocidade de dissipação térmica (edifícios mais isolados têm $\alpha$ menor), enquanto $\beta$ representa a eficiência de regulação térmica do HVAC.


%------------------------------------------------------------------------------------

\subsubsection*{Consumo Energético do HVAC}

O consumo energético é proporcional ao esforço necessário para manter o conforto térmico.  
Quanto maior a diferença entre a temperatura exterior e a de conforto, maior é a energia necessária para compensar as perdas ou ganhos térmicos.  
A equação seguinte define este comportamento:

\[
P_{\text{HVAC}} = \max \Big(20,\; 50 + 5 \cdot |T_{\text{conforto}} - T_{\text{exterior}}| + 20 \cdot |T_{\text{interior}} - T_{\text{conforto}}| + \eta \Big)
\]

\textbf{Onde:}
\begin{itemize}[nosep]
	\item $P_{\text{HVAC}}$: consumo energético instantâneo (W);
	\item $|T_{\text{conforto}} - T_{\text{exterior}}|$: esforço devido às condições exteriores;
	\item $|T_{\text{interior}} - T_{\text{conforto}}|$: esforço interno para manter o conforto;
	\item $\eta$: ruído aleatório para simular variações de carga;
	\item $\max(20, \cdot)$: garante um consumo mínimo de 20~W (modo de \textit{standby}). \\
\end{itemize}

Este modelo cria uma relação linear simplificada entre o desvio térmico e o consumo energético.

\subsection{Transformação e Enriquecimento (Pentaho Kettle)}

Os dados simulados no Node-RED são posteriormente processados no Pentaho.  
O objetivo é normalizar os registos, limpar os nomes e derivar métricas adicionais de desempenho térmico.

\begin{itemize}
	\item \textbf{CSV Input:} leitura dos dados gerados pelo Node-RED;
	\item \textbf{Regex Replace:} limpeza e padronização de nomes de dispositivos e salas;
	\item \textbf{Calculator:} criação de variáveis derivadas para análise de eficiência:
	\[
	\text{temp\_diff} = |T_{\text{exterior}} - 21|
	\quad\text{e}\quad
	\text{efficiency\_index} = \frac{\text{avg\_power}}{(\text{temp\_diff} + 1)}
	\]
	Estas métricas permitem avaliar o impacto da diferença térmica no consumo e comparar o desempenho entre divisões.
	\item \textbf{CSV/XML Output:} exportação dos dados tratados e enriquecidos.
\end{itemize}

O resultado é um conjunto de ficheiros estruturados prontos para análise estatística e visualização em Python.


%------------------------------------------------------------------------------------

\newpage
\section{Transformações e Jobs (Pentaho)}

\subsection{Transformações}

A transformação principal, designada \textbf{``etl\_climate\_analysis''}, foi desenvolvida no Pentaho Data Integration (Kettle) e tem como objetivo processar os ficheiros CSV gerados pelo Node-RED, realizar operações de limpeza e enriquecimento, e exportar os resultados para múltiplos destinos.\\

A transformação é composta pelas seguintes etapas principais:

\begin{itemize}
	\item \textbf{CSV File Input:}  
	Lê os dados provenientes do Node-RED. Cada linha contém o timestamp, o identificador do dispositivo, o nome da divisão, o tipo de sensor, o valor medido e a temperatura exterior correspondente.  
	Os campos são importados com os respetivos tipos (string, number, date).
	
	\item \textbf{Validate Timestamp (Regex):}  
	Validação da coluna de data/hora através de uma expressão regular, garantindo o formato correto antes do processamento. Linhas com timestamps inválidos são rejeitadas.
	
	\item \textbf{Filter Rows (Sensor Type = Power):}  
	Filtra apenas os registos relativos ao consumo energético (\texttt{sensor\_type = "power"}), eliminando medições de temperatura interna.  
	Este passo reduz o volume de dados e foca a análise no comportamento energético do sistema HVAC.
	
	\item \textbf{Group By (Aggregate Power):}  
	Os dados são agregados por \textbf{timestamp} e \textbf{divisão}, calculando:
	\begin{itemize}
		\item \texttt{avg\_power} — consumo médio de energia;
		\item \texttt{outdoor\_temp} — média da temperatura exterior;
	\end{itemize}
	
	\item \textbf{Formula Step (Calculated Fields):}  
	Através da etapa \textit{Formula}, são geradas novas colunas derivadas:
	\begin{itemize}
		\item \texttt{efficiency} — eficiência térmica estimada com base no rácio entre potência média e diferença térmica;
		\item \texttt{temp\_diff} — diferença absoluta entre temperatura interior simulada e exterior.
	\end{itemize}
	Estes cálculos complementam a análise energética do sistema HVAC.
	
	\item \textbf{Table Output (SQLite):}  
	Os dados enriquecidos são inseridos numa base de dados SQLite local (\texttt{hvac\_data.db}).
	
	\item \textbf{Text File Output (CSV Export):}  
	Exporta o dataset final para formato CSV, permitindo integração com outras ferramentas analíticas.
	
	\item \textbf{XML Output:}  
	Gera um ficheiro XML com a mesma informação, facilitando a interoperabilidade e validação estrutural dos dados.
\end{itemize}

O fluxo completo garante a consistência entre diferentes formatos de exportação e cria um dataset limpo e consolidado para análise posterior.

\subsection{Jobs}

O job, denominado \textbf{``etl\_hvac\_power''}, orquestra a execução das transformações e gere o fluxo global de execução.

As etapas principais são:

\begin{itemize}
	\item \textbf{Log Start:}  
	Gera uma entrada de log com a data/hora de início, permitindo o acompanhamento das execuções no histórico de logs.
	
	\item \textbf{Run Transformation (climate\_transform):}  
	Chama a transformação principal, realizando todo o processo de ETL descrito anteriormente.  
	Caso ocorram erros, o job é interrompido e o resultado é registado como falha.
	
	\item \textbf{Log End:}  
	Regista a conclusão da execução e o estado final (sucesso ou erro), encerrando o processo de forma controlada.
	
	\item \textbf{Email:}
	Manda um email para confirmar o fim do job com os logs.
\end{itemize}

Este job garante a execução automatizada e sequencial das tarefas.

%------------------------------------------------------------------------------------

\newpage
\section{Análise e Visualização (Python)}

Após a execução do pipeline ETL, os ficheiros finais são processados em Python para análise estatística e visualização gráfica dos dados.  
As bibliotecas utilizadas incluem:

\begin{itemize}
	\item \texttt{pandas} (v$\geq$2.0) para manipulação, limpeza, agregação e fusão de datasets;
	\item \texttt{matplotlib} (v$\geq$3.7) para a criação de gráficos básicos de linhas, barras e dispersão;
	\item \texttt{seaborn} (v$\geq$0.13) para visualizações estatísticas avançadas, incluindo boxplots e regressões lineares.
\end{itemize}

Foram produzidos os seguintes gráficos a partir do dataset processado:

\begin{itemize}
	\item \textbf{Scatter Plot — Power vs Outdoor Temperature:} visualiza a relação entre a temperatura exterior e o consumo médio de energia (\texttt{avg\_power}). (Figura: \texttt{power\_vs\_outdoor\_temp.png})
	\item \textbf{Scatter Plot — Power vs Temperature Difference:} mostra a relação entre a diferença de temperatura (\texttt{temp\_diff}) e o consumo médio de energia. (Figura: \texttt{power\_vs\_temp\_diff.png})
	\item \textbf{3D Scatter Plot — Outdoor Temp vs Temp Diff vs Avg Power:} fornece uma visão tridimensional do consumo energético em função da temperatura exterior e da diferença de temperatura, codificando a eficiência pelo esquema de cores. (Figura: \texttt{3d\_power\_temp.png})
	\item \textbf{Histogram of Efficiency:} distribuição da eficiência do HVAC ao longo de todos os registos, permitindo identificar padrões e outliers. (Figura: \texttt{efficiency\_hist.png})
	\item \textbf{Boxplot of Avg Power per Room:} análise comparativa do consumo médio de energia por sala, destacando variações entre divisões. (Figura: \texttt{avg\_power\_boxplot.png})
\end{itemize}

Estas visualizações permitem identificar padrões de consumo, períodos de maior esforço do HVAC, diferenças entre salas e a relação entre condições exteriores e desempenho energético.  
A utilização conjunta de scatter plots, 3D plots, histogramas e boxplots fornece uma análise completa e facilmente interpretável do comportamento do sistema HVAC, complementando os cálculos realizados no Pentaho.

%------------------------------------------------------------------------------------

\newpage
\section{Vídeo de Demonstração}

Para complementar o relatório, foi produzido um vídeo demonstrativo do pipeline completo, incluindo:

\begin{itemize}
	\item Simulação de dados com Node-RED;
	\item Processamento e transformação no Pentaho (Kettle);
	\item Exportação de dados para CSV, XML e SQLite;
	\item Visualização e análise em Python.
\end{itemize}

O vídeo pode ser acedido diretamente através do seguinte QR Code:

\begin{center}
	\includegraphics[width=0.4\textwidth]{qrcode_video.png}
\end{center}

\noindent \textbf{Link direto:} \url{https://drive.google.com/file/d/1q7uXCaNz5Eu7ypg1x0YI0lOxh1eN9QmP/view}

%------------------------------------------------------------------------------------

\newpage
\section{Conclusão e Trabalhos Futuros}

O presente trabalho permitiu desenvolver e implementar um pipeline completo de ETL para monitorização e análise de sistemas HVAC, desde a recolha de dados até à visualização de resultados.  
Foram alcançados os seguintes objetivos:

\begin{itemize}
	\item Integração de dados simulados e reais via Node-RED, combinando informação de sensores internos e dados meteorológicos externos;
	\item Processamento e limpeza de dados no Pentaho, incluindo validação de timestamps, filtragem, agregações e cálculo de métricas derivadas (\texttt{efficiency} e \texttt{temp\_diff});
	\item Exportação de dados para vários formatos (CSV, XML e SQLite) garantindo persistência e facilidade de análise;
	\item Visualização e análise estatística em Python, permitindo identificar padrões de consumo energético, comportamento térmico das divisões e eficiência do sistema HVAC.
\end{itemize}

O pipeline demonstrou ser reprodutível, modular e escalável, fornecendo \textit{insights} relevantes para monitorização e otimização energética de edifícios.

\textbf{Trabalhos Futuros:}

\begin{itemize}
	\item Implementação de logging mais detalhado e métricas de desempenho do ETL, permitindo monitorização contínua do pipeline;
	\item Expansão da simulação para incluir outros parâmetros ambientais, como humidade, ocupação e consumo real de energia;
\end{itemize}

Estas melhorias permitiriam transformar o pipeline atual num sistema inteligente de monitorização e otimização de climatização, apto a utilização em contextos reais de edifícios residenciais ou comerciais.

%----------------------------------------------------------------------------------------
%	BIBLIOGRAPHY
%----------------------------------------------------------------------------------------
\nocite{*}
\newpage
\printbibliography
%----------------------------------------------------------------------------------------

\end{document}