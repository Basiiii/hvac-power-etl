%----------------------------------------------------------------------------------------
%	PACKAGES AND OTHER DOCUMENT CONFIGURATIONS
%----------------------------------------------------------------------------------------

\documentclass[a4paper, 12pt]{article} % Font size (can be 10pt, 11pt or 12pt) and paper size (remove a4paper for US letter paper)
\usepackage[portuguese]{babel}
\usepackage{amsmath}
\usepackage[protrusion=true,expansion=true]{microtype} % Better typography
\usepackage{graphicx} % Required for including pictures
\usepackage{wrapfig} % Allows in-line images
\usepackage{pdflscape} %Allows landscape oriented pages
\usepackage{ulem} % para sublinhar
\usepackage[colorlinks=true, urlcolor=blue]{hyperref}%Required for hyperlink references
\hypersetup{
	colorlinks=true, % Ativa links coloridos em vez de caixas ao redor
	linkcolor=black, % Cor dos links internos
	citecolor=black, % Cor das citações
	urlcolor=black   % Cor dos links externos (Jira, por exemplo)
}

\usepackage{mathpazo} % Use the Palatino font
\usepackage[T1]{fontenc} % Required for accented characters
\linespread{1.05} % Change line spacing here, Palatino benefits from a slight increase by default
\usepackage{float}
\usepackage[backend=bibtex,style=numeric]{biblatex}
\addbibresource{references.bib} % Nome do arquivo de referências
\makeatletter
\renewcommand\@biblabel[1]{\textbf{#1.}} % Change the square brackets for each bibliography item from '[1]' to '1.'
\renewcommand{\@listI}{\itemsep=0pt} % Reduce the space between items in the itemize and enumerate environments and the bibliography

\usepackage{amsmath, amssymb}
\usepackage{graphicx}
\usepackage{hyperref}
\usepackage{geometry}
\usepackage{enumitem}
\usepackage{qrcode}

\renewcommand{\maketitle}{
\begin{titlepage}
\begin{center}
\vspace*{1cm}
%\includegraphics[width=0.35\textwidth]{../images/logo-no-bg.png}\\[1cm] % Logo
{\Huge\textbf{Clima e Consumo HVAC}}\\[0.5cm] % Main Title
{\Large Integração de Sistemas de Informação}\\[2cm] % Subtitle
{\large \textsc{
	Enrique Rodrigues Nº28602}}\\[0.5cm] % Author
{\textit{Instituto Politécnico do Cávado e do Ave}}\\[1.5cm] % Institution
{\large \today} % Date
\vfill
\textbf{Palavras-chave:} Node-RED, Pentaho Kettle, Integração de Dados, IoT, Simulação, HVAC, Modelação Térmica, ETL, Python, pandas, matplotlib
\end{center}
\end{titlepage}
}
\makeatother

%------------------------------------------------------------------------------------
\begin{document}
\maketitle % Print the title section

%----------------------------------------------------------------------------------------
%	ABSTRACT AND KEYWORDS
%----------------------------------------------------------------------------------------

\renewcommand{\abstractname}{Summary} % Uncomment to change the name of the abstract to something else

% \begin{abstract}
% Morbi tempor congue porta. Proin semper, leo vitae faucibus dictum, metus mauris lacinia lorem, ac congue leo felis eu turpis. Sed nec nunc pellentesque, gravida eros at, porttitor ipsum. Praesent consequat urna a lacus lobortis ultrices eget ac metus. In tempus hendrerit rhoncus. Mauris dignissim turpis id sollicitudin lacinia. Praesent libero tellus, fringilla nec ullamcorper at, ultrices id nulla. Phasellus placerat a tellus a malesuada.
% \end{abstract}

% \hspace*{3,6mm}\textit{Keywords:} lorem , ipsum , dolor , sit amet , lectus % Keywords

% \vspace{30pt} % Some vertical space between the abstract and first section

%----------------------------------------------------------------------------------------
%	DOCUMENT BODY
%----------------------------------------------------------------------------------------

\newpage
\renewcommand{\contentsname}{Índice}
\tableofcontents

%------------------------------------------------------------------------------------

%\newpage
%\renewcommand{\listfigurename}{Lista de Figuras}
%\listoffigures

%------------------------------------------------------------------------------------

\newpage
\section{Enquadramento}

O presente trabalho descreve o desenvolvimento de um pipeline de dados completo para recolha, transformação e análise de variáveis ambientais, nomeadamente:
temperatura exterior, temperatura interior simulada e consumo energético de um sistema HVAC.

O sistema é composto por três módulos principais:
\begin{itemize}
	\item \textbf{Node-RED:} aquisição de dados e simulação das variáveis ambientais;
	\item \textbf{Pentaho Data Integration (Kettle):} transformação, limpeza, enriquecimento e exportação dos dados;
	\item \textbf{Python (pandas, matplotlib, seaborn):} análise estatística e visualização gráfica dos resultados.
\end{itemize}

%------------------------------------------------------------------------------------

\newpage
\section{Problema}

A gestão eficiente da climatização em edifícios residenciais e comerciais é um desafio crescente, dada a variabilidade das condições exteriores e o impacto direto no consumo energético.  
O controlo manual do HVAC ou a falta de monitorização contínua leva a desperdício de energia e a condições de conforto inconsistentes.\\

O objetivo deste trabalho é demonstrar um pipeline de dados capaz de:
\begin{itemize}
	\item Recolher temperaturas exteriores via API pública, permitindo monitorização;
	\item Simular a evolução da temperatura interior e o consumo do HVAC, com base em modelos térmicos iterativos;
	\item Processar e enriquecer os dados no Pentaho;
	\item Exportar resultados em CSV e XML para análise posterior;
	\item Gerar gráficos em Python que permitam avaliar padrões de consumo e identificar oportunidades de otimização energética.
\end{itemize}

Desta forma, o pipeline não só replica o comportamento térmico de um edifício como também fornece dados estruturados que permitem tomar decisões informadas sobre a eficiência do sistema HVAC.

%------------------------------------------------------------------------------------

\newpage
\section{Estratégia Utilizada}

\subsection{Aquisição e Simulação (Node-RED)}

Os dados de temperatura exterior são obtidos via API REST.  
A simulação da temperatura interior e do consumo do HVAC é realizada em JavaScript.
Cada sala tem parâmetros térmicos próprios (\(\alpha, \beta\)).
\subsection{Modelação Matemática das Variáveis Simuladas}

Esta secção descreve os modelos utilizados para simular a evolução térmica das divisões e o consumo energético do sistema HVAC.  
Os modelos foram implementados em JavaScript, correndo em ciclo a partir da temperatura anterior e das condições exteriores obtidas por API.  
Os parâmetros $\alpha$ e $\beta$ variam por divisão e representam, respetivamente, o grau de isolamento e a capacidade de resposta do sistema de climatização.

\subsubsection*{Evolução da Temperatura Interior}

A temperatura interior de cada divisão evolui de forma gradual, aproximando-se da temperatura exterior devido às trocas térmicas do edifício e da temperatura de conforto devido à atuação do HVAC.
A equação seguinte modela esta dinâmica:

\[
T_{\text{interior}}(t) = T_{\text{anterior}} + \alpha \cdot (T_{\text{exterior}} - T_{\text{anterior}}) + \beta \cdot (T_{\text{conforto}} - T_{\text{anterior}}) + \epsilon
\]

\textbf{Onde:}
\begin{itemize}[nosep]
	\item $T_{\text{interior}}(t)$: temperatura interior no instante atual;
	\item $T_{\text{anterior}}$: temperatura interior no instante anterior;
	\item $T_{\text{exterior}}$: temperatura exterior (obtida por API);
	\item $T_{\text{conforto}}$: temperatura de conforto (21°C);
	\item $\alpha$: coeficiente de isolamento térmico do edifício;
	\item $\beta$: coeficiente de correção aplicado pelo HVAC;
	\item $\epsilon$: ruído aleatório que introduz variação realista.\\
\end{itemize}

Este modelo é iterativo, a cada ciclo de simulação, a temperatura interior é atualizada em função dos valores anteriores.  
A componente $\alpha$ controla a velocidade de dissipação térmica (edifícios mais isolados têm $\alpha$ menor), enquanto $\beta$ representa a eficiência de regulação térmica do HVAC.

\subsubsection*{Consumo Energético do HVAC}

O consumo energético é proporcional ao esforço necessário para manter o conforto térmico.  
Quanto maior a diferença entre a temperatura exterior e a de conforto, maior é a energia necessária para compensar as perdas ou ganhos térmicos.  
A equação seguinte define este comportamento:

\[
P_{\text{HVAC}} = \max \Big(20,\; 50 + 5 \cdot |T_{\text{conforto}} - T_{\text{exterior}}| + 20 \cdot |T_{\text{interior}} - T_{\text{conforto}}| + \eta \Big)
\]

\textbf{Onde:}
\begin{itemize}[nosep]
	\item $P_{\text{HVAC}}$: consumo energético instantâneo (W);
	\item $|T_{\text{conforto}} - T_{\text{exterior}}|$: esforço devido às condições exteriores;
	\item $|T_{\text{interior}} - T_{\text{conforto}}|$: esforço interno para manter o conforto;
	\item $\eta$: ruído aleatório para simular variações de carga;
	\item $\max(20, \cdot)$: garante um consumo mínimo de 20~W (modo de \textit{standby}). \\
\end{itemize}

Este modelo cria uma relação linear simplificada entre o desvio térmico e o consumo energético.

\subsection{Transformação e Enriquecimento (Pentaho Kettle)}

Os dados simulados no Node-RED são posteriormente processados no Pentaho.  
O objetivo é normalizar os registos, limpar os nomes e derivar métricas adicionais de desempenho térmico.

\begin{itemize}
	\item \textbf{CSV Input:} leitura dos dados gerados pelo Node-RED;
	\item \textbf{Regex Replace:} limpeza e padronização de nomes de dispositivos e salas;
	\item \textbf{Calculator:} criação de variáveis derivadas para análise de eficiência:
	\[
	\text{temp\_diff} = |T_{\text{exterior}} - 21|
	\quad\text{e}\quad
	\text{efficiency\_index} = \frac{\text{avg\_power}}{(\text{temp\_diff} + 1)}
	\]
	Estas métricas permitem avaliar o impacto da diferença térmica no consumo e comparar o desempenho entre divisões.
	\item \textbf{CSV/XML Output:} exportação dos dados tratados e enriquecidos.
\end{itemize}

O resultado é um conjunto de ficheiros estruturados prontos para análise estatística e visualização em Python.


%------------------------------------------------------------------------------------

\newpage
\section{Transformações e Jobs (Pentaho)}

\subsection{Transformações}
%\begin{center}
%	\includegraphics[width=0.8\textwidth]{transformacao_pentaho.png}
%\end{center}
A transformação inclui leitura, limpeza e criação de novas colunas.  
Os campos são normalizados e exportados em dois formatos.

\subsection{Jobs}
%\begin{center}
%	\includegraphics[width=0.8\textwidth]{job_pentaho.png}
%\end{center}
O job executa automaticamente as transformações, verifica a existência dos ficheiros de entrada e gera logs de execução.

%------------------------------------------------------------------------------------

\newpage
\section{Análise e Visualização (Python)}

Os ficheiros finais são processados em Python usando:
\begin{itemize}
	\item \texttt{pandas} (v$\geq$2.0) para manipulação e agregação de dados;
	\item \texttt{matplotlib} (v$\geq$3.7) para gráficos base;
	\item \texttt{seaborn} (v$\geq$0.13) para visualizações estatísticas.
\end{itemize}

Foram produzidos gráficos de temperatura exterior/interior e consumo energético médio por sala e por dia.

%------------------------------------------------------------------------------------

\newpage
\section{Vídeo com Demonstração}
%\begin{center}
%	\qrcode{https://linkparaoseuvideo.com}
%\end{center}

%------------------------------------------------------------------------------------

\newpage
\section{Conclusão e Trabalhos Futuros}

O pipeline integra aquisição, simulação, transformação e visualização.  
A abordagem modular permite fácil expansão e automação.  

\textbf{Trabalhos futuros:}
\begin{itemize}
	\item Integração direta Node-RED \(\rightarrow\) Pentaho via API;
	\item Dashboards interativos (Grafana, Power BI);
	\item Modelos preditivos de consumo energético.
\end{itemize}

%----------------------------------------------------------------------------------------
%	BIBLIOGRAPHY
%----------------------------------------------------------------------------------------
\nocite{*}

% \printbibliography
%----------------------------------------------------------------------------------------

\end{document}